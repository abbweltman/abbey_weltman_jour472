% Options for packages loaded elsewhere
\PassOptionsToPackage{unicode}{hyperref}
\PassOptionsToPackage{hyphens}{url}
\PassOptionsToPackage{dvipsnames,svgnames,x11names}{xcolor}
%
\documentclass[
  letterpaper,
  DIV=11,
  numbers=noendperiod]{scrartcl}

\usepackage{amsmath,amssymb}
\usepackage{iftex}
\ifPDFTeX
  \usepackage[T1]{fontenc}
  \usepackage[utf8]{inputenc}
  \usepackage{textcomp} % provide euro and other symbols
\else % if luatex or xetex
  \usepackage{unicode-math}
  \defaultfontfeatures{Scale=MatchLowercase}
  \defaultfontfeatures[\rmfamily]{Ligatures=TeX,Scale=1}
\fi
\usepackage{lmodern}
\ifPDFTeX\else  
    % xetex/luatex font selection
\fi
% Use upquote if available, for straight quotes in verbatim environments
\IfFileExists{upquote.sty}{\usepackage{upquote}}{}
\IfFileExists{microtype.sty}{% use microtype if available
  \usepackage[]{microtype}
  \UseMicrotypeSet[protrusion]{basicmath} % disable protrusion for tt fonts
}{}
\makeatletter
\@ifundefined{KOMAClassName}{% if non-KOMA class
  \IfFileExists{parskip.sty}{%
    \usepackage{parskip}
  }{% else
    \setlength{\parindent}{0pt}
    \setlength{\parskip}{6pt plus 2pt minus 1pt}}
}{% if KOMA class
  \KOMAoptions{parskip=half}}
\makeatother
\usepackage{xcolor}
\setlength{\emergencystretch}{3em} % prevent overfull lines
\setcounter{secnumdepth}{-\maxdimen} % remove section numbering
% Make \paragraph and \subparagraph free-standing
\ifx\paragraph\undefined\else
  \let\oldparagraph\paragraph
  \renewcommand{\paragraph}[1]{\oldparagraph{#1}\mbox{}}
\fi
\ifx\subparagraph\undefined\else
  \let\oldsubparagraph\subparagraph
  \renewcommand{\subparagraph}[1]{\oldsubparagraph{#1}\mbox{}}
\fi

\usepackage{color}
\usepackage{fancyvrb}
\newcommand{\VerbBar}{|}
\newcommand{\VERB}{\Verb[commandchars=\\\{\}]}
\DefineVerbatimEnvironment{Highlighting}{Verbatim}{commandchars=\\\{\}}
% Add ',fontsize=\small' for more characters per line
\usepackage{framed}
\definecolor{shadecolor}{RGB}{241,243,245}
\newenvironment{Shaded}{\begin{snugshade}}{\end{snugshade}}
\newcommand{\AlertTok}[1]{\textcolor[rgb]{0.68,0.00,0.00}{#1}}
\newcommand{\AnnotationTok}[1]{\textcolor[rgb]{0.37,0.37,0.37}{#1}}
\newcommand{\AttributeTok}[1]{\textcolor[rgb]{0.40,0.45,0.13}{#1}}
\newcommand{\BaseNTok}[1]{\textcolor[rgb]{0.68,0.00,0.00}{#1}}
\newcommand{\BuiltInTok}[1]{\textcolor[rgb]{0.00,0.23,0.31}{#1}}
\newcommand{\CharTok}[1]{\textcolor[rgb]{0.13,0.47,0.30}{#1}}
\newcommand{\CommentTok}[1]{\textcolor[rgb]{0.37,0.37,0.37}{#1}}
\newcommand{\CommentVarTok}[1]{\textcolor[rgb]{0.37,0.37,0.37}{\textit{#1}}}
\newcommand{\ConstantTok}[1]{\textcolor[rgb]{0.56,0.35,0.01}{#1}}
\newcommand{\ControlFlowTok}[1]{\textcolor[rgb]{0.00,0.23,0.31}{#1}}
\newcommand{\DataTypeTok}[1]{\textcolor[rgb]{0.68,0.00,0.00}{#1}}
\newcommand{\DecValTok}[1]{\textcolor[rgb]{0.68,0.00,0.00}{#1}}
\newcommand{\DocumentationTok}[1]{\textcolor[rgb]{0.37,0.37,0.37}{\textit{#1}}}
\newcommand{\ErrorTok}[1]{\textcolor[rgb]{0.68,0.00,0.00}{#1}}
\newcommand{\ExtensionTok}[1]{\textcolor[rgb]{0.00,0.23,0.31}{#1}}
\newcommand{\FloatTok}[1]{\textcolor[rgb]{0.68,0.00,0.00}{#1}}
\newcommand{\FunctionTok}[1]{\textcolor[rgb]{0.28,0.35,0.67}{#1}}
\newcommand{\ImportTok}[1]{\textcolor[rgb]{0.00,0.46,0.62}{#1}}
\newcommand{\InformationTok}[1]{\textcolor[rgb]{0.37,0.37,0.37}{#1}}
\newcommand{\KeywordTok}[1]{\textcolor[rgb]{0.00,0.23,0.31}{#1}}
\newcommand{\NormalTok}[1]{\textcolor[rgb]{0.00,0.23,0.31}{#1}}
\newcommand{\OperatorTok}[1]{\textcolor[rgb]{0.37,0.37,0.37}{#1}}
\newcommand{\OtherTok}[1]{\textcolor[rgb]{0.00,0.23,0.31}{#1}}
\newcommand{\PreprocessorTok}[1]{\textcolor[rgb]{0.68,0.00,0.00}{#1}}
\newcommand{\RegionMarkerTok}[1]{\textcolor[rgb]{0.00,0.23,0.31}{#1}}
\newcommand{\SpecialCharTok}[1]{\textcolor[rgb]{0.37,0.37,0.37}{#1}}
\newcommand{\SpecialStringTok}[1]{\textcolor[rgb]{0.13,0.47,0.30}{#1}}
\newcommand{\StringTok}[1]{\textcolor[rgb]{0.13,0.47,0.30}{#1}}
\newcommand{\VariableTok}[1]{\textcolor[rgb]{0.07,0.07,0.07}{#1}}
\newcommand{\VerbatimStringTok}[1]{\textcolor[rgb]{0.13,0.47,0.30}{#1}}
\newcommand{\WarningTok}[1]{\textcolor[rgb]{0.37,0.37,0.37}{\textit{#1}}}

\providecommand{\tightlist}{%
  \setlength{\itemsep}{0pt}\setlength{\parskip}{0pt}}\usepackage{longtable,booktabs,array}
\usepackage{calc} % for calculating minipage widths
% Correct order of tables after \paragraph or \subparagraph
\usepackage{etoolbox}
\makeatletter
\patchcmd\longtable{\par}{\if@noskipsec\mbox{}\fi\par}{}{}
\makeatother
% Allow footnotes in longtable head/foot
\IfFileExists{footnotehyper.sty}{\usepackage{footnotehyper}}{\usepackage{footnote}}
\makesavenoteenv{longtable}
\usepackage{graphicx}
\makeatletter
\def\maxwidth{\ifdim\Gin@nat@width>\linewidth\linewidth\else\Gin@nat@width\fi}
\def\maxheight{\ifdim\Gin@nat@height>\textheight\textheight\else\Gin@nat@height\fi}
\makeatother
% Scale images if necessary, so that they will not overflow the page
% margins by default, and it is still possible to overwrite the defaults
% using explicit options in \includegraphics[width, height, ...]{}
\setkeys{Gin}{width=\maxwidth,height=\maxheight,keepaspectratio}
% Set default figure placement to htbp
\makeatletter
\def\fps@figure{htbp}
\makeatother

\KOMAoption{captions}{tableheading}
\makeatletter
\makeatother
\makeatletter
\makeatother
\makeatletter
\@ifpackageloaded{caption}{}{\usepackage{caption}}
\AtBeginDocument{%
\ifdefined\contentsname
  \renewcommand*\contentsname{Table of contents}
\else
  \newcommand\contentsname{Table of contents}
\fi
\ifdefined\listfigurename
  \renewcommand*\listfigurename{List of Figures}
\else
  \newcommand\listfigurename{List of Figures}
\fi
\ifdefined\listtablename
  \renewcommand*\listtablename{List of Tables}
\else
  \newcommand\listtablename{List of Tables}
\fi
\ifdefined\figurename
  \renewcommand*\figurename{Figure}
\else
  \newcommand\figurename{Figure}
\fi
\ifdefined\tablename
  \renewcommand*\tablename{Table}
\else
  \newcommand\tablename{Table}
\fi
}
\@ifpackageloaded{float}{}{\usepackage{float}}
\floatstyle{ruled}
\@ifundefined{c@chapter}{\newfloat{codelisting}{h}{lop}}{\newfloat{codelisting}{h}{lop}[chapter]}
\floatname{codelisting}{Listing}
\newcommand*\listoflistings{\listof{codelisting}{List of Listings}}
\makeatother
\makeatletter
\@ifpackageloaded{caption}{}{\usepackage{caption}}
\@ifpackageloaded{subcaption}{}{\usepackage{subcaption}}
\makeatother
\makeatletter
\@ifpackageloaded{tcolorbox}{}{\usepackage[skins,breakable]{tcolorbox}}
\makeatother
\makeatletter
\@ifundefined{shadecolor}{\definecolor{shadecolor}{rgb}{.97, .97, .97}}
\makeatother
\makeatletter
\makeatother
\makeatletter
\makeatother
\ifLuaTeX
  \usepackage{selnolig}  % disable illegal ligatures
\fi
\IfFileExists{bookmark.sty}{\usepackage{bookmark}}{\usepackage{hyperref}}
\IfFileExists{xurl.sty}{\usepackage{xurl}}{} % add URL line breaks if available
\urlstyle{same} % disable monospaced font for URLs
\hypersetup{
  pdftitle={abbey\_data\_analysis},
  colorlinks=true,
  linkcolor={blue},
  filecolor={Maroon},
  citecolor={Blue},
  urlcolor={Blue},
  pdfcreator={LaTeX via pandoc}}

\title{abbey\_data\_analysis}
\author{}
\date{}

\begin{document}
\maketitle
\ifdefined\Shaded\renewenvironment{Shaded}{\begin{tcolorbox}[boxrule=0pt, interior hidden, borderline west={3pt}{0pt}{shadecolor}, breakable, enhanced, sharp corners, frame hidden]}{\end{tcolorbox}}\fi

\begin{Shaded}
\begin{Highlighting}[]
\FunctionTok{library}\NormalTok{(tidyverse)}
\end{Highlighting}
\end{Shaded}

\begin{verbatim}
-- Attaching core tidyverse packages ------------------------ tidyverse 2.0.0 --
v dplyr     1.1.4     v readr     2.1.5
v forcats   1.0.0     v stringr   1.5.1
v ggplot2   3.4.4     v tibble    3.2.1
v lubridate 1.9.3     v tidyr     1.3.1
v purrr     1.0.2     
-- Conflicts ------------------------------------------ tidyverse_conflicts() --
x dplyr::filter() masks stats::filter()
x dplyr::lag()    masks stats::lag()
i Use the conflicted package (<http://conflicted.r-lib.org/>) to force all conflicts to become errors
\end{verbatim}

\begin{Shaded}
\begin{Highlighting}[]
\FunctionTok{library}\NormalTok{(janitor)}
\end{Highlighting}
\end{Shaded}

\begin{verbatim}

Attaching package: 'janitor'

The following objects are masked from 'package:stats':

    chisq.test, fisher.test
\end{verbatim}

\begin{Shaded}
\begin{Highlighting}[]
\FunctionTok{library}\NormalTok{(sf)}
\end{Highlighting}
\end{Shaded}

\begin{verbatim}
Linking to GEOS 3.11.0, GDAL 3.5.3, PROJ 9.1.0; sf_use_s2() is TRUE
\end{verbatim}

\begin{Shaded}
\begin{Highlighting}[]
\FunctionTok{library}\NormalTok{(leaflet)}
\FunctionTok{library}\NormalTok{(rvest)}
\end{Highlighting}
\end{Shaded}

\begin{verbatim}

Attaching package: 'rvest'

The following object is masked from 'package:readr':

    guess_encoding
\end{verbatim}

\begin{Shaded}
\begin{Highlighting}[]
\FunctionTok{library}\NormalTok{(ggplot2)}
\FunctionTok{library}\NormalTok{(dplyr)}
\FunctionTok{library}\NormalTok{(htmltools)}
\end{Highlighting}
\end{Shaded}

\textbf{Exploring Business Development in Baltimore}

In this notebook I will look into the dynamics of business development
in Baltimore, Maryland. By examining data sets related to the
distribution of new businesses across different regions of Baltimore, I
aim to uncover insights into the patterns and trends of entrepreneurial
activity within the city.

My analysis focuses on several key aspects:

\textbf{Spatial Distribution of New Businesses:} We investigate which
regions of Baltimore have the highest concentration of new businesses,
providing valuable insights into areas experiencing significant
entrepreneurial activity.

\textbf{Percentage of New Businesses Relative to Total:} By comparing
the percentage of new businesses to the total number of businesses in
each region, we gain a deeper understanding of the prevalence of
entrepreneurial ventures within the broader business landscape of
Baltimore.

\textbf{Analysis of Business Development:} I will explore how business
development has evolved over recent years by examining changes in the
percentage of new businesses from previous years to the present,
shedding light on the dynamics of economic growth and adaptation.

\textbf{Mapping Business Development Trends} to represent the spatial
distribution of new businesses and highlight regions experiencing
notable shifts in entrepreneurial activity. All the data I used in my
project came from the Baltimore Neighborhood Indicators Alliance. They
have a data set that showed the total businesses in each community
statistical area and the percent of businesses in each community
statistical area that were under 4 years old. I also used a similar
dataset for percent of businesses in each community statistical area
that were under one year old and the geoid tracts of Baltimore. Then I
used the Baltimore mapping resources to identify the CSA polygons and
create a map.

\textbf{Data Used:}

\textbf{``Total Number of Businesses''} 1.
https://vital-signs-bniajfi.hub.arcgis.com/datasets/bniajfi::total-number-of-businesses/explore?layer=0\&location=39.284764\%2C-76.620440\%2C11.72

\textbf{``Percent of Businesses that are Four Years Old or Less''} 2.
https://vital-signs-bniajfi.hub.arcgis.com/datasets/bniajfi::percent-of-businesses-that-are-4-years-old-or-less/explore?layer=0\&location=39.284764\%2C-76.620440\%2C11.72

\textbf{``Percent of Businesses that are One Year Old or Less''} 3.
https://vital-signs-bniajfi.hub.arcgis.com/datasets/bniajfi::percent-of-businesses-that-are-1-year-old-or-less-2/explore?layer=0\&location=39.284764\%2C-76.620440\%2C11.72

\textbf{``Community Statistical Area 2020''} 4.
https://mapping-bniajfi.opendata.arcgis.com/datasets/9be011c4ed2b481ab83e4f2cf2a04b78\_0/explore?location=39.284764\%2C-76.620524\%2C11.72

\begin{Shaded}
\begin{Highlighting}[]
\CommentTok{\#First I am going to load my datasets }
\NormalTok{new\_business }\OtherTok{\textless{}{-}} \FunctionTok{read\_csv}\NormalTok{(}\StringTok{"Percent\_of\_Businesses\_that\_are\_4\_Years\_old\_or\_less.csv"}\NormalTok{)}
\end{Highlighting}
\end{Shaded}

\begin{verbatim}
Rows: 55 Columns: 17
-- Column specification --------------------------------------------------------
Delimiter: ","
chr  (2): CSA2010, CSA2020
dbl (15): OBJECTID, biz4_10, biz4_11, biz4_12, biz4_13, biz4_14, biz4_15, bi...

i Use `spec()` to retrieve the full column specification for this data.
i Specify the column types or set `show_col_types = FALSE` to quiet this message.
\end{verbatim}

\begin{Shaded}
\begin{Highlighting}[]
\NormalTok{total\_businesses }\OtherTok{\textless{}{-}} \FunctionTok{read\_csv}\NormalTok{(}\StringTok{"Total\_Number\_of\_Businesses.csv"}\NormalTok{)}
\end{Highlighting}
\end{Shaded}

\begin{verbatim}
Rows: 55 Columns: 17
-- Column specification --------------------------------------------------------
Delimiter: ","
chr  (2): CSA2010, CSA2020
dbl (15): OBJECTID, numbus10, numbus11, numbus12, numbus13, numbus14, numbus...

i Use `spec()` to retrieve the full column specification for this data.
i Specify the column types or set `show_col_types = FALSE` to quiet this message.
\end{verbatim}

\begin{Shaded}
\begin{Highlighting}[]
\NormalTok{one\_yr\_biz }\OtherTok{\textless{}{-}} \FunctionTok{read\_csv}\NormalTok{(}\StringTok{"Percent\_of\_Businesses\_that\_are\_1\_Year\_old\_or\_less (1).csv"}\NormalTok{)}
\end{Highlighting}
\end{Shaded}

\begin{verbatim}
Rows: 55 Columns: 17
-- Column specification --------------------------------------------------------
Delimiter: ","
chr  (2): CSA2010, CSA2020
dbl (15): OBJECTID, biz1_10, biz1_11, biz1_12, biz1_13, biz1_14, biz1_15, bi...

i Use `spec()` to retrieve the full column specification for this data.
i Specify the column types or set `show_col_types = FALSE` to quiet this message.
\end{verbatim}

\begin{Shaded}
\begin{Highlighting}[]
\NormalTok{Census\_Tract\_2020 }\OtherTok{\textless{}{-}} \FunctionTok{read\_csv}\NormalTok{(}\StringTok{"Census\_Tract\_(2020)\_to\_Community\_Statistical\_Area\_(2020).csv"}\NormalTok{)}
\end{Highlighting}
\end{Shaded}

\begin{verbatim}
Rows: 199 Columns: 4
-- Column specification --------------------------------------------------------
Delimiter: ","
chr (2): Tract_2020, Community_Statistical_Area_2020
dbl (2): GEOID_Tract_2020, ObjectId

i Use `spec()` to retrieve the full column specification for this data.
i Specify the column types or set `show_col_types = FALSE` to quiet this message.
\end{verbatim}

\begin{Shaded}
\begin{Highlighting}[]
\NormalTok{CSAGEO }\OtherTok{\textless{}{-}} \FunctionTok{read\_csv}\NormalTok{(}\StringTok{"CSAGEO {-} Sheet1.csv"}\NormalTok{)}
\end{Highlighting}
\end{Shaded}

\begin{verbatim}
Rows: 56 Columns: 4
-- Column specification --------------------------------------------------------
Delimiter: ","
chr (1): CSA2020
dbl (1): SQMI
num (2): Shape__Area, Shape__Length

i Use `spec()` to retrieve the full column specification for this data.
i Specify the column types or set `show_col_types = FALSE` to quiet this message.
\end{verbatim}

\begin{Shaded}
\begin{Highlighting}[]
\CommentTok{\#the first dataset I want to look at is which regions in Baltimore have the most business under 4 years old. I am going to use the average of two most recent years. }

\NormalTok{new\_business\_high }\OtherTok{\textless{}{-}}\NormalTok{ new\_business  }\SpecialCharTok{\%\textgreater{}\%}
 \FunctionTok{select}\NormalTok{(CSA2020,biz4\_20, biz4\_21) }\SpecialCharTok{\%\textgreater{}\%}
 \FunctionTok{arrange}\NormalTok{(}\FunctionTok{desc}\NormalTok{(biz4\_21))}\SpecialCharTok{\%\textgreater{}\%}
\FunctionTok{mutate}\NormalTok{(}\AttributeTok{biz\_av=} \FunctionTok{round}\NormalTok{((biz4\_20 }\SpecialCharTok{+}\NormalTok{ biz4\_21)}\SpecialCharTok{/}\DecValTok{2}\NormalTok{,}\DecValTok{2}\NormalTok{))}
 \CommentTok{\#sort(desc("biz\_av"))}

\CommentTok{\#The region of Baltimore with highest percent of new businesses from 2021 was Oldtown/Middle East. }

\NormalTok{new\_business\_low }\OtherTok{\textless{}{-}}\NormalTok{ new\_business  }\SpecialCharTok{\%\textgreater{}\%}
 \FunctionTok{select}\NormalTok{(CSA2020,biz4\_20, biz4\_21) }\SpecialCharTok{\%\textgreater{}\%}
  \FunctionTok{arrange}\NormalTok{(biz4\_21)}\SpecialCharTok{\%\textgreater{}\%}
  \FunctionTok{mutate}\NormalTok{(}\AttributeTok{biz\_av=} \FunctionTok{round}\NormalTok{((biz4\_20 }\SpecialCharTok{+}\NormalTok{ biz4\_21)}\SpecialCharTok{/}\DecValTok{2}\NormalTok{,}\DecValTok{2}\NormalTok{))}\SpecialCharTok{\%\textgreater{}\%} 
\FunctionTok{head}\NormalTok{(}\DecValTok{10}\NormalTok{)}
\CommentTok{\#The region of Baltimore with lowest percent of new businesses from 2021 was Westport/Mount Winans/Lakeland. This is interesting also becasue these cities are less then 10 miles away from eachother but have such different entreprenuiral culture. }
\end{Highlighting}
\end{Shaded}

\begin{Shaded}
\begin{Highlighting}[]
\CommentTok{\#this graphic shows the areas with the most business development in Baltimore. It shows how much higher Oldtown/Middle East is from the rest.}
\NormalTok{iframe\_code }\OtherTok{\textless{}{-}} \StringTok{\textquotesingle{}}
\StringTok{\textless{}iframe title="Business Development In Baltimore" aria{-}label="Bar Chart" id="datawrapper{-}chart{-}0eJiM" src="https://datawrapper.dwcdn.net/0eJiM/1/" scrolling="no" frameborder="0" style="border: none;" width="600" height="1409" data{-}external="1"\textgreater{}\textless{}/iframe\textgreater{}}
\StringTok{\textquotesingle{}}
\CommentTok{\# Embed the iframe code into an HTML object}
\NormalTok{iframe\_html }\OtherTok{\textless{}{-}} \FunctionTok{HTML}\NormalTok{(iframe\_code)}

\CommentTok{\# Display the HTML object}
\NormalTok{iframe\_html}
\end{Highlighting}
\end{Shaded}

\begin{Shaded}
\begin{Highlighting}[]
\CommentTok{\#The next question I wanted to solve was what percentage of total businesses in a region of Baltimore are under 4 Years old. To determine this I will combine the total business table with the businessmen under 4 years old table to compare the percents of many business in the area are relatively new. }

\NormalTok{pct\_total\_new\_biz }\OtherTok{\textless{}{-}}\NormalTok{ total\_businesses }\SpecialCharTok{\%\textgreater{}\%}
\FunctionTok{inner\_join}\NormalTok{(new\_business, }\AttributeTok{by=}\StringTok{"CSA2020"}\NormalTok{)}\SpecialCharTok{\%\textgreater{}\%}
\FunctionTok{select}\NormalTok{(CSA2020, numbus21, biz4\_21)}\SpecialCharTok{\%\textgreater{}\%}
\FunctionTok{mutate}\NormalTok{(}\AttributeTok{pct\_whole =} \FunctionTok{round}\NormalTok{((biz4\_21 }\SpecialCharTok{/}\NormalTok{ numbus21) }\SpecialCharTok{*} \DecValTok{100}\NormalTok{,}\DecValTok{2}\NormalTok{))}\SpecialCharTok{\%\textgreater{}\%}
\FunctionTok{arrange}\NormalTok{(}\FunctionTok{desc}\NormalTok{(pct\_whole))}

\CommentTok{\#Edmondson Village had the highest percent of 4 or less year old businesses out of the regions total amount of businesses. The percent of new businesses out of the total is about 55.5\% }
\end{Highlighting}
\end{Shaded}

\begin{Shaded}
\begin{Highlighting}[]
\CommentTok{\#the next question I wanted to determine was how did new business development grow from 2019{-}2021. I will determine this by comparing the values of new business under 4 years old for the past three years and then determine the percent change. then I can filter to see which 10 regions had the highest change in percentage points. }

\NormalTok{biz\_dev }\OtherTok{\textless{}{-}}\NormalTok{ new\_business }\SpecialCharTok{\%\textgreater{}\%}
\FunctionTok{select}\NormalTok{(CSA2020, biz4\_19, biz4\_21) }\SpecialCharTok{\%\textgreater{}\%}
\FunctionTok{mutate}\NormalTok{(}\AttributeTok{pct\_change =} \FunctionTok{round}\NormalTok{((biz4\_21 }\SpecialCharTok{{-}}\NormalTok{ biz4\_19) }\SpecialCharTok{/}\NormalTok{ biz4\_19 }\SpecialCharTok{*} \DecValTok{100}\NormalTok{,}\DecValTok{2}\NormalTok{))}\SpecialCharTok{\%\textgreater{}\%}
\FunctionTok{arrange}\NormalTok{(}\FunctionTok{desc}\NormalTok{(pct\_change))}

\CommentTok{\#The region of Baltimore with the largest percentage change from 2019{-}2021 in business development is Dickeyville/Franklintown with a percent change of 112\%. }
\end{Highlighting}
\end{Shaded}

\begin{Shaded}
\begin{Highlighting}[]
\CommentTok{\#the question I want to look at the business development in Baltimore under one year old. To do this I am going to select the top 10 regions of Baltimore that had the highest percent of business under 1 year old.}

\NormalTok{one\_yr\_constant }\OtherTok{\textless{}{-}}\NormalTok{ one\_yr\_biz }\SpecialCharTok{\%\textgreater{}\%}
  \FunctionTok{inner\_join}\NormalTok{(new\_business, }\AttributeTok{by =} \StringTok{"CSA2020"}\NormalTok{) }\SpecialCharTok{\%\textgreater{}\%}
\FunctionTok{select}\NormalTok{(CSA2020, biz1\_21) }\SpecialCharTok{\%\textgreater{}\%}
\FunctionTok{arrange}\NormalTok{(}\FunctionTok{desc}\NormalTok{(biz1\_21))}

\CommentTok{\#the region of Baltimore with the highest percent of business 1 year old or less is  Oldtown/Middle East this is consistent with what I saw in the the 4 years or less data. So there are consistently experiencing rapid business development over the past 4 years so now I want to see the percent change of just Oldtown/Middle East in the past couple years. }

\NormalTok{oldtown\_pctchange }\OtherTok{\textless{}{-}}\NormalTok{ one\_yr\_biz }\SpecialCharTok{\%\textgreater{}\%}
\FunctionTok{inner\_join}\NormalTok{(new\_business, }\AttributeTok{by =} \StringTok{"CSA2020"}\NormalTok{) }\SpecialCharTok{\%\textgreater{}\%}
  \FunctionTok{select}\NormalTok{(CSA2020, biz1\_19, biz1\_21) }\SpecialCharTok{\%\textgreater{}\%}
  \FunctionTok{filter}\NormalTok{(CSA2020 }\SpecialCharTok{==} \StringTok{"Oldtown/Middle East"}\NormalTok{) }\SpecialCharTok{\%\textgreater{}\%}
\FunctionTok{mutate}\NormalTok{(}\AttributeTok{pct\_change =}\NormalTok{ ((biz1\_21 }\SpecialCharTok{{-}}\NormalTok{ biz1\_19) }\SpecialCharTok{/}\NormalTok{ biz1\_19 }\SpecialCharTok{*} \DecValTok{100}\NormalTok{))}

\CommentTok{\# The Percent change from 2019{-}2021 in Oldtown/ Middle East Region of Baltimore was {-}56\%. This likely due to the pandemic. }
\end{Highlighting}
\end{Shaded}

\textbf{Final Memo:}

The analysis of business development in Baltimore reveals insights into
the city's entrepreneurial landscape.

Firstly, the spatial distribution of new businesses unveils regional
hotspots of entrepreneurial activity, with areas like Oldtown/Middle
East yeilding a high concentration of new businesses. On the other hand,
regions like Westport/Mount Winans/Lakeland exhibit lower levels of
entrepreneurial growth.

Examining the percentage of new businesses relative to the total number
in each region provides a understanding of the prevalence of new
businesses within Baltimore's business landscape. For instance,
Edmondson Village emerges as a standout area with approximately 55.5\%
of its businesses being under four years old, reflecting significant
entrepreneurial environment.

Furthermore, analyzing the evolution of business development from 2019
to 2021 sheds light on the dynamic changes in entrepreneurial activity.
Dickeyville/Franklintown stands out with a remarkable 112\% increase in
business development during this period, indicating notable growth and
adaptation within the local business ecosystem.

Mapping these trends visually shows the spatial distribution of new
businesses and highlights regions experiencing notable shifts in
entrepreneurial activity. This visual representation provides
stakeholders with valuable insights into the changing dynamics of
business development across Baltimore.

Overall, these findings suggest a dynamic and evolving entrepreneurial
landscape in Baltimore, with certain regions experiencing rapid growth
while others navigate challenges such as the impact of the COVID-19
pandemic. Such insights can inform strategic decision-making and policy
interventions aimed at fostering sustainable economic growth and
prosperity in the city.

ADD PITCH



\end{document}
